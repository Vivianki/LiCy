% ---
% Arquivo com a conclusão do Trabalho de Conclusão de Curso dos alunos
% Gabriel Takaoka Nishimura, Felippe Demarqui Ramos e Vivian Kimie Isuyama 
% da Escola Politécnica da Universidade de São Paulo
% ---
	\chapter{Conclusão}\label{cap-conclusao}
		
	Os principais desafios encontrados podem ser divididos entre as duas frentes de desenvolvimento. No caso do hardware digital, podem-se apontar como desafios significativos a compreensão e implantação dos algoritmos teóricos de correção de erros em VHDL, bem como manter sincronicidade entre diferentes subcomponentes da FPGA. Uma dificuldade especialmente relevante foi integrar os componentes dentro de um pipeline, com a presença de circuitos de controle para tomar decisões sobre um circuito de fluxo de dados, já que a depuração é dificultada pelas ferramentas de desenvolvimento. Por conseguinte, pode-se ressaltar ainda que o funcionamento teórico de circuitos no Quartus não garante automaticamente o mesmo comportamento dentro de uma memória de fato. Portanto, a transição entre os dois ambientes levou um tempo significativo para depuração e ajustes. Finalmente, possíveis aperfeiçoamentos identificados incluem otimizações nos circuitos digitais para aumentar a vazão de dados efetiva, já que a própria camada física PHY I limita a taxa de transmissão para valores relativamente baixos se observadas as camadas PHY II e PHY III, o que pode igualmente limitar as aplicações finais do Li-Fi.
	
	Por outro lado, o hardware analógico também ocasionou demandas muito expressivas dentro do projeto. Conforme o que se discutiu na sessão de execução, houve demasiados incrementos nos circuitos para lidar com eliminação de sinais parasitários, filtragem da luz ambiente e o trabalho adequado na frequência de operação indicada pela norma para a camada implementada. Com efeito, ainda existem possíveis melhorias caso este estudo seja levado à frente. Atualmente, o transmissor e receptor devem estar a até uma distância máxima para que a comunicação se estabeleça. É desejável que se aumente essa distância máxima através de uma abordagem analógica.
	
	Apesar de ter uma taxa de comunicação reduzida, a camada PHY I mostrou-se muito eficaz para manter a integridade dos dados, sendo recomendada para aplicações onde esse fator é importante. Sendo assim, a norma IEEE 802.15.7 foi um fator decisivo para mitigar a existência de erros e possibilitar uma comunicação estável entre dois pontos, dadas certas condições como distância máxima e a ausência de obstáculos entre transmissor e receptor. Portanto, este projeto é um protótipo indicativo de que futuramente a disruptividade da comunicação por luz visível tem potencial para ultrapassar do campo teórico e tornar-se realidade em nível de manufatura e produto. 
	
	\section{Trabalhos Futuros}
	
	A partir da análise anterior e dos resultados citados, podem-se determinar alguns passos para trabalhos futuros. Entre eles, faz-se necessária a integração com camadas mais altas de comunicação, ressaltada a MAC, indicada pela norma. Uma alternativa para a implantação desta camada, que não foi discutida neste estudo, seria o uso de um microcontrolador, de modo a ser um intermediário entre a camada física e camadas mais altas, não especificadas na norma. O uso das camadas físicas mais avançadas, PHY II e PHY III também é desejável, já que elas aumentam a vazão dos dados e diminuem possíveis interferências, estendendo a gama de aplicações para a comunicação Li-Fi.
	
	Ademais, é ainda importante para a continuidade do projeto encontrar aplicações pertinentes ao usuário do produto. Possivelmente, um uso interessante seria acoplar o módulo receptor a um celular ou dispositivo móvel, o que demandaria usar uma interface USB e desenvolver um aplicativo para realizar requisições e mostrar informações. Finalmente, num cenário futuro, seria possível conectar um celular à internet usando a tecnologia Li-Fi, portanto com o estabelecimento de bilateralidade na comunicação.
