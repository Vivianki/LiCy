% ---
% Arquivo com os resumos do Trabalho de Conclusão de Curso dos alunos
% Gabriel Takaoka Nishimura, Felippe Demarqui Ramos e Vivian Kimie Isuyama 
% da Escola Politécnica da Universidade de São Paulo
% ---

% resumo em português
\setlength{\absparsep}{18pt} % ajusta o espaçamento dos parágrafos do resumo
\begin{resumo}
	O presente trabalho analisa a comunicação por luz visível, dentro do contexto da norma IEEE 802.15.7. Verificaram-se, pois, alternativas para uma implantação da primeira camada física definida nesta norma, ressaltados os mecanismos de codificação, decodificação e correção de erros, transmissão e recepção pela luz visível. A partir desta análise foi definido um projeto, cuja execução final gerou um módulo receptor e um módulo transmissor, entre os quais acontece transferência de dados pela luz, sem presença de cabeamento entre os dois. Com efeito, os resultados finais indicam a viabilidade desse tipo de comunicação para projetos futuros, com destaque para aplicações no uso doméstico ou privado. 
	
	\vspace{\onelineskip}
	\noindent 
	\textbf{Palavras-chave}: comunicação por luz visível. Li-Fi. IEEE 802.15.7. PHY I.
\end{resumo}

% resumo em inglês
\begin{resumo}[Abstract]
	\begin{otherlanguage*}{english}
		The current project analyses the concept of visible light communication within the context of the IEEE 802.15.7 standard. The main activities included the assessment of alternatives for the deployment of the first physical layer defined by this standard. Therefore, methods concerning encoding, decoding, error correction, transmission and reception through visible light were emphasized. Through this evaluation the project was defined. Its final execution has generated reception and transmission units, which are able to transfer data without any wiring system. Effectively, the final deliverables point out the viability of this type of communication for future projects, highlighting domestic and private applications.
		
		\vspace{\onelineskip}
		\noindent 
		\textbf{Keywords}: visual light communication. Li-Fi. IEEE 802.15.7. PHY I.
	\end{otherlanguage*}
\end{resumo}